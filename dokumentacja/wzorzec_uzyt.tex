\section{Dokumentacja użytkowa}

\begin{center}
\textit{Czyli co użytkownik powinien wiedzieć}
\end{center}



\subsection{Czego użytkownik powinien oczekiwać?}
Prostej aplikacji, która pozwoli mu utworzyć własną sieć komputerową. Użytkownik powinien przygotować się na używanie linii komend. Aplikacja, zainstalowana na danym komputerze, przyłączy komputer do sieci lub też, jeśli taka jest wola użytkownika, utworzy całkowicie nową sieć.

\subsection{Tworzenie wykresu}
Aby umożliwić łatwy sposób na zobaczenie aktualnego wyglądu sieci program automatycznie, po dołączeniu każdego nowego komputera, generuje na wszystkich maszynach w sieci wykresy obrazujące połączenia między komputerami.\\
\indent W celu obejrzenia wykresu należy otworzyć plik $chart.html$ znajdujący się w katalogu $src/resources$.

\subsection{Przykładowy wygląd wykresu}

\includegraphics[width=1.0\textwidth]{chart.jpg}

\subsection{Jak uruchomić program?}
Program został skonstruowany w taki sposób aby nawet komputerowmu laikowi zapewnić łatwy dostęp do jego funkcjonalności. Aby skorzystać z możliwości oferowanych przez program należy uruchomić plik $main$, znajdujący się w katalogu $src$, z prawami administratora.\\
\indent Po uruchomieniu program będzie wymagał od użytkownika wprowadzenia następujących danych:
\begin{enumerate}
\item Adres IP komputera na którym jest uruchomiony program.
\item Liczba dozwolonych połączeń nawiązanych przez komputer na którym został uruchomiony program - dzięki temu komputery o lepszych zdolnościach obliczeniowych mogą być bardziej obciążane od słabszych komputerów.
\item Czy jest to pierwszy komputer w sieci - odpowiedzią mogą być: yes/y/no/n, w przypadku odpowiedzi twierdzącej komputer tworzy nową sieć do której można podłączać inne maszyny.
\item Jeśli odpowiedź na poprzednie pytanie była przecząca, program pyta o adres IP dowolnego komputera należącego do sieci docelowej - jest to potrzebne w celu umożliwienia jakiejkolwiek komunikacji komputera z siecią.
\end{enumerate}
\indent I to już wszystko! Po wprowadzeniu powyższych informacji program już sam się wszystkim zajmie, więc użytkownik nie musi się niczym przejmować.

\subsection{Podsumowanie}
Jesteśmy pewni, iż dzięki powyższej instrukcji nawet Użytkownicy niezwiązani z branżą komputerową poradzą sobie z utworzeniem sieci komputerowej w swojej firmie!\\
\indent Tych bardziej zorientowanych w temacie oraz ciekawych szczegółów technicznych zapraszamy do zapoznania się z dokumentacją implementacyjną oraz techniczną.