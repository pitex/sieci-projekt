\section{Dokumentacja techniczna}

\begin{center}
\textit{Czyli szczegóły techniczne}
\end{center}

\subsection{Klient - serwer}
Podstawową jednostką naszej sieci jest klient, czyli użytkownik. Klientem takim jest nowo dołączany komputer. Najważniejszą funkcjonalnością sieci jest komunikacja klienta z serwerem.\\
\indent Za obsługę tejże komunikacji odpowiedzialny jest package joinservice, a konkretnie - znajdujące się w nim implementacje client.go i server.go.\\
\indent Techniczne szczegóły dotyczące zamieszczonych w nich konstruktorów i metod znnajdują się w komentarzach, a zatem również w wygenerowanym docu. Komentarze są na tyle dokładne, ze nie wymagają dodatkowego tłumaczenia - zapraszamy do ich lektury!

\subsection{Drzewo}
Funkcjonalności naszej sieci opierają się na strukturze drzewa, które tworzy. Odpowiada za nie joinservice/tree.\\
\indent Klasa node.go trzyma reprezentację drzewa jako drzewa wskaźnikowego - każdy wierzchołek ma listę wskaźników do swoich dzieci.\\
\indent Funkcje DFS oraz FindSolution przeszukują drzewo metodą DFS na potrzeby obu funkcjonalności, które oferujemy.\\
\indent Dlatego obsługę drzewa zdecydowaliśmy się umieścić w pakiecie z klientem i serwerem. Jest to klasa kluczowa dla działania całej algorytmiki w aplikacji.

\subsection{Protokoły}
...znajdują się w pakiecie protocols. Opis ich działania zamieściliśmy w dokumentacji implementacyjnej.\\
\indent Z kodu w oczywisty sposób wynikają uzasadnienia poszczególnych naszych decyzji.\\
\indent Funkcje znajdujące się w protokołach to głównie funkcje pomocnicze służące do interpretacji, parse'owania i obsługi danych.\\
\indent Za budowę wiadomości odpowiada struct Message (w każdym protokole z osobna). W przypadku STP jest niezwykle prosty. Sposób, w jaki używamy go w SIP jasno wynika z komentarzy. Są w nich wyszczególnione możliwe typy wiadomości - są one kluczowe w naszej aplikacji, warto się z nimi zapoznać!

\subsection{Pakiet resources}
\indent odpowiada za skrypt tworzenia grafu. Jest w niego wpisany kod skryptu, komputery budują odpowiedni plik łącząc pliki odpowiadające za początek i koniec skryptu.\\
\indent Zdecydowaliśmy się na takie rozwiązanie, ponieważ musimy wpisać "środek skryptu", który jest zmienny i odpowiada za kształt wykresu (patrz klasa node.go - jego funkcja DFS wpisuje do odpowiedniego pliku tenże "środek").

\subsection{Szczegółowy opis działania wszystkich funkcji}
...znajduje się w anglojęzycznym docu: \url{http://godoc.org/github.com/pitex/sieci-projekt/src}\\
\indent Opisy są bardzo dokładne - zapraszamy do lektury!
