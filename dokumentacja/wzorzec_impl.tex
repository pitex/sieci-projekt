\section{Dokumentacja implementacyjna}

\begin{center}
\textit{Czyli szczegóły implementacyjne}
\end{center}


\subsection{Algorymiczna idea rozwiązania}
Tworzymy sieć w formie ukorzenionego drzewa (korzeń stanowi pierwsza maszyna). Nowy komputer dołączamy do takiego wierzchołka, który nie osiągnął maksymalnej pojemności i ma najniższą głębokość. Jeśli takich wierzchołków jest więcej, wybieramy ten najbardziej na lewo.

\begin{enumerate}
\item % 1
Nowo dodany komputer połączony będzie z co najwyżej jednym komputerem dodanym wcześniej. Dlatego nigdy nie powstanie nam cykl i sieć zachowa strukturę drzewa.
\item % 2
Każde drzewo ma liście. Pojemność wierzchołków jest większa od 2, zatem zawsze istnieje wierzchołek, do którego możemy coś dodać.
\item %3
Zauważmy, że średnica powiększa się jedynie wtedy, gdy tworzymy nowy poziom w drzewie. Nasze postępowanie prowadzi jednak do tego, by nowy poziom był tworzony jak najrzadziej.
\end{enumerate}
Oczywiście, globalnie rozwiązanie to nie jest optymalne, ale my musimy radzić sobie z problemem on-line!


\subsection{Utrudnienie i rozwiązanie tego problemu}
JEDNAKŻE, POJAWIA SIĘ PEWIEN PROBLEM: Co będzie, jeśli dostaniemy dwa requesty o dołączenie do sieci na raz?
Możemy rozwiązać go w następujący sposób: root będzie znał strukturę drzewa. To on będzie znajdował maszynę, do której podpięty zostanie nowo dodany komputer.


\subsection{Podstawowe dane na temat specyfikacji projektu}
Zatem... jak się do tego zabrać technicznie? Na początku dokonaliśmy następujących wyborów:\\
\\
WYBRANY JĘZYK: go\\
\\
POŁĄCZENIE MIĘDZY KOMUPUTERAMI: socket\\
\\
POZOSTAŁE TECHNOLOGIE: Google Charts (https://developers.google.com/chart/)\\
\\

\subsection{Dlaczego golang?}
Go oferuje łatwość tworzenia aplikacji sieciowych oraz wydajność. Aplikacja napisana w pythonie byłaby podobnie skomplikowana, natomiast mało wydajna. Tworzenie aplikacji sieciowych w C++ jest na tyle utrudnione, że nie chcieliśmy się w nie zagłębiać.

\subsection{Dlaczego Google charts?}
Jest to prosta, przydatna technologia, którą warto znać. Pozwala na tworzenie nawet bardzo skomplikowanych wykresów stosunkowo niewielkim kosztem - należy znać jedynie podstawy JavaScript (a właściwie - schemat HTML/CSS/JAVASCRIPT). Duża ilość dostępnych na stronie przykładów pozwala każdemu zagłębić się w tę technologię. Dlatego uznaliśmy, że jest ona idealna dla tworzenia wykresu naszej sieci.

\subsection{Szczegóły rozwiązania: dodawanie nowego komputera}
Informację o tym, że przychodzi nowy komputer, otrzymuje dowolna maszyna w naszej sieci. Następnie każdy wysyła zapytanie o wynik do swojego rodzica, w końcu dojdzie ono do korzenia.\\
Korzeń znajduje wierzchołek, do którego należy podpiąć nową maszynę na wspomniany wcześniej sposób. Można do tego użyć zwyczajnego algorytmu DFS.\\
Rozsyła później informację do wszystkich swoich dzieci, które przekazują ją dalej. Wiadomość rozprzestrzenia się w całej sieci.


\subsection{Szczegóły rozwiązania: pierwotna idea na tworzenie wykresu}
wierzchołek otrzymuje polecenie "zbuduj wykres". Wysyła wówczas do swojego rodzica tę wiadomość. Gdy wiadomość dochodzi do korzenia, tworzy on skrypt budujący wykres oraz rozsyła go do swoich dzieci z komunikatem mówiącym ZBUDOWAŁEM SKRYPT. Zostaje ona przekazana dalej.\\

\subsection{Szczegóły rozwiązania: problem z tworzeniem wykresu i jego rozwiązanie}
Jednakże, dziecko nie wie, czy zmiany w drzewie zaszły, zatem przy częstych prośbach o wykres nasza sieć zostanie zasypana niepotrzebnymi danymi.\\Zatem nowy wykres będzie tworzony razem z dodaniem nowego wierzchołka i wówczas rozsyłany, co oznacza, że każda maszyna trzyma sobie update'owany wykres.

\subsection{Wybrane protokoły}

PROTOKÓŁ WARSTWY TRANSPORTOWEJ: tcp. Nie zależy nam na szybkości, ponieważ mamy do czynienia z małą, wewnętrzną siecią. Natomiast przy tworzeniu sieci bardzo ważne są dla nas bezpieczeństwo danych oraz dokładność.\\
Poniżej opisane są protokoły warstwy aplikacji.

\subsection{Protokół SIP}

PROTOKÓŁ 1 - PRZESYŁANIE INFORMACJI MIĘDZY KOMPUTERAMI (podczas tworzenia sieci, przekazywania informacji o klientach, "dogadywania się" w sprawie przesłania skryptu wykresu): wystarczy bardzo prosty protokół tekstowy. Jedyne informacje, jakie potrzebujemy, to: typ wiadomości, jej treść oraz ewentualna informacja o błędzie. Zauważmy, że jest to wiadomość, która jest bardzo niewielka, nie potrzebujemy zatem w żaden sposób "porcjować" danych. Będzie ona przekazywana w taki sposób, że wysyłamy tę samą wiadomość póki nie otrzymamy informacji zwrotnej - potwierdzenia jej otrzymania. \\
Opis uzytych przez nas wiadomości oraz format danych znajduje się w dokumentacji.\\
\\
Protokołowi temu zdecydowaliśmy nadać nazwę Simple Information Protocol - w skrócie SIP.

\subsection{Protokół STP}
PROTOKÓŁ 2 - PRZESYŁANIE WYKRESU: plik z wykresem może być duży, musimy zatem przesyłać go w mniejszych pakietach. Możemy użyć protokołu tekstowego i przekazać po prostu kod wykresu w JavaScripcie. Zauważmy, że komputery mogą ustalić między sobą fakt, że zaraz nastąpi tranfser danych za pomocą naszego protokołu. Następnie możemy podzielić przesyłany skrypt na odpowiednio małe fragmenty i przesyłać je do momentu otrzymania informacji zwrotnej na ich temat.\\
\\
Protokół ten nazwaliśmy Script Transfet Protocol - STP.

\subsection{Krótkie uzasadnienie}

Dlaczego zamiast używania SIP oraz STP nie użyjemy jednego protokołu, o formacie jak SIP, skoro równie dobrze można wysyłać podzielony plik za pomocą SIP? Zwróćmy uwagę na to, ile niepotrzebnych informacji zostałoby wówczas przekazane i jak bardzo spowolniłoby to transfer danych!

\subsection{Bezpieczeństwo}
Tworzona sieć jest na użytek wewnętrzny, zatem uznaliśmy, że nie musimy czynić dodatkowych wysiłków w celu gwarancji jej większego bezpieczeństwa. Jest ono częściowo zapewnione poprzez używanie sieci lokalnej. Aplikacja ma na celu utworzenie sieci bardzo prostej, służącej do wewnętrznej komunikacji pracowników, nie do przesyłania danych poufnych, które w wypadku firm niewielkich mogą znajdować się poza siecią. W wypadku, gdy Użytkownik będzie chciał przesyłać dane o znaczących rozmiarach czy dane o szczególnej poufności należałoby dorobić przynajmniej o jeden protokół więcej, nie posiadamy bowiem protokołu typu FTP.\\
Jednakże, projektując aplikację zadbaliśmy o wprowadzenie metodyki Agile - "programowanie zwinne". Zakłada ona częste zmiany preferencji użytkowników, a co za tym idzie, wymaga łatwości w przeprowadzaniu modyfikacji kodu. Daje to możliwość dodania zabezpieczeń w wypadku dalszego rozwoju naszej aplikacji. \\
Możliwości, jeśli chodzi o wprowadzenie bezpieczeństwa, są niezliczone - pierwszą, dość oczywistą, jest zabezpieczenie sieci hasłem. Każdy komputer w sieci posiadałby hash hasła, przed dołączeniem do sieci rodzic wymagałby od nowego dziecka podania hasła.
