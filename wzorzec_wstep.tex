\section{Wstęp}

\begin{center}
\textit{Czyli o co chodzi w naszym projekcie?}
\end{center}


\subsection{Po co taki projekt?}

W dzisiejszych czasach młoda firma, która chce prężnie się rozwijać, często potrzebuje sieć komputerową na własne, wewnętrzne potrzeby.
Postanowiliśmy stworzyć aplikację, która pomoże rozwiązać ten problem.
Dzięki naszej aplikacji Użytkownik, który dokładnie i ze zrozumieniem przeczyta instrukcję, może samodzielnie zbudować prostą, ale niezwykle wydajną sieć komputerową na potrzeby swojej firmy.
\\Ma to dla Użytkownika duże znaczenie - początkujące, młode firmy niekoniecznie mogą pozwolić sobie na zatrudnienie specjalisty.
\\Dla ułatwienia Użytkownikowi pracy i zrozumienia działania aplikacji, wyposażyliśmy ją w dodatkową funkcjonalność - tworzenie wykresu sieci komputerowej. Dzięki niej Użytkownik pozna strukturę sieci, stanie się ona również niezwykłą pomocą w razie awarii sieci.

\subsection{Naukowe określenie problemu}

Tworzymy sieć komputerową o strukturze drzewowej, dla każdej maszyny istnieje z góry określona maksymalna pojemność - ilość maszyn, z którymi może być połączona (zakładamy, że jest to liczba większa od 2). Chcemy zachować najmniejszą możliwą średnicę sieci, tj. chcemy, aby odległość między najbardziej oddalonymi wierzchołkami była jak najmniejsza. Dochodzi nowy komputer, który musimy dodać on-line. Co robimy?
